Tipos estruturados:
\begin{parts}
	\part[2] Descreva as vantagens da utilização de tipos estruturados.
	\part[2] Declare uma estrutura para representar uma Data. A estrutura deve conter os campos dia, mês e ano.
	\part[2] Declare uma estrutura para representar uma Atividade. A estrutura deverá conter os campos título, descrição, nota e Data de entrega. 
	\part[2] Implemente uma função que receba (por referência) uma Atividade e (por valor) a Data de hoje. 
A função deverá alterar a nota da atividade para metade se a atividade estiver atrasada 
e imprimir a mensagem ``Entregue com atraso''. 
Caso contrário imprimir a mensagem ``Entregue no prazo''. 
A função deverá obedecer a assinatura: \\\\
\texttt{void valida (Atividade * entregue, Data hoje);} \\
	\part[2] Implemente uma função que receba um vetor de Atividades e a quantidade de Atividades realizadas. 
A função deverá calcular e retornar a nota média das Atividades (considere que todas as Atividades possuem o mesmo peso).
A função deverá obedecer a assinatura:\\\\
\texttt{float media (Atividade atividades[], int quantidade);}\\
\end{parts}

