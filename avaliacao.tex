\documentclass[a4paper, 11pt, addpoints]{exam} % Adicione o parametro answers para ver as respostas

% Pacote que contém a configuracao do modelo
\usepackage{UniAval}

\begin{document}

% Define a configuracao do cabecalho
\nomeUniversidade{Universidade do Estado de Santa Catarina}
\logoUniversidade{fig/udesc}
\escalaLogoUniversidade{0.35}
\nomeCurso{Centro de Educação do Planalto Norte - CEPLAN}
\nomeProfessor{Nome do Professor}
\nomeDisciplina{Nome da Disciplina}
\dataDaProva{14/05/2018}
\siglaRegistroAcademico{Matrícula}

% Mostra o cabecalho
\info

% Mostra as questoes existentes e quanto o aluno acertou em cada uma
\pontuacao

%%%%%%%%%%%%%%%%%%%%%%%%%%%%%%%%%%%%%%%%%%%%%%%%%

%%AAG: tipos de questoes:
%  - \begin{checkboxes}
%  - \begin{choices}
%  - \begin{oneparchoices}
%  - \begin{oneparcheckboxes}
% vide exam.cls

% Documentação do exam.cls: http://www-math.mit.edu/~psh/exam/examdoc.pdf

\vspace{5pt}
    
\begin{questions}
	\question[10] \input{questoes/vetores/testemesa1.tex} 
	\question[5] \input{questoes/vetores/instrucoes1.tex} 
	\question[5] \input{questoes/alocacao_dinamica/teorica1.tex} 
	\question \input{questoes/struct/questao3.tex}
	\question \input{questoes/lista/questao1.tex}
	\question \input{questoes/estruturasdedados/questao9.tex}
\end{questions}

%Mensagem ao final da avaliacao
% \begin{bottompar}
% 	{\bf Porque esse é meu caminho ninja!}
% 	\includegraphics[width=0.1\textwidth]{fig/aldeia.png}
% \end{bottompar}

\end{document}
